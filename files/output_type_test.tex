% Options for packages loaded elsewhere
\PassOptionsToPackage{unicode}{hyperref}
\PassOptionsToPackage{hyphens}{url}
%
\documentclass[
]{article}
\usepackage{amsmath,amssymb}
\usepackage{lmodern}
\usepackage{ifxetex,ifluatex}
\ifnum 0\ifxetex 1\fi\ifluatex 1\fi=0 % if pdftex
  \usepackage[T1]{fontenc}
  \usepackage[utf8]{inputenc}
  \usepackage{textcomp} % provide euro and other symbols
\else % if luatex or xetex
  \usepackage{unicode-math}
  \defaultfontfeatures{Scale=MatchLowercase}
  \defaultfontfeatures[\rmfamily]{Ligatures=TeX,Scale=1}
\fi
% Use upquote if available, for straight quotes in verbatim environments
\IfFileExists{upquote.sty}{\usepackage{upquote}}{}
\IfFileExists{microtype.sty}{% use microtype if available
  \usepackage[]{microtype}
  \UseMicrotypeSet[protrusion]{basicmath} % disable protrusion for tt fonts
}{}
\makeatletter
\@ifundefined{KOMAClassName}{% if non-KOMA class
  \IfFileExists{parskip.sty}{%
    \usepackage{parskip}
  }{% else
    \setlength{\parindent}{0pt}
    \setlength{\parskip}{6pt plus 2pt minus 1pt}}
}{% if KOMA class
  \KOMAoptions{parskip=half}}
\makeatother
\usepackage{xcolor}
\IfFileExists{xurl.sty}{\usepackage{xurl}}{} % add URL line breaks if available
\IfFileExists{bookmark.sty}{\usepackage{bookmark}}{\usepackage{hyperref}}
\hypersetup{
  pdftitle={R Markdown Output Type Test},
  pdfauthor={R. M. D. Trickster},
  hidelinks,
  pdfcreator={LaTeX via pandoc}}
\urlstyle{same} % disable monospaced font for URLs
\usepackage[margin=1in]{geometry}
\usepackage{longtable,booktabs,array}
\usepackage{calc} % for calculating minipage widths
% Correct order of tables after \paragraph or \subparagraph
\usepackage{etoolbox}
\makeatletter
\patchcmd\longtable{\par}{\if@noskipsec\mbox{}\fi\par}{}{}
\makeatother
% Allow footnotes in longtable head/foot
\IfFileExists{footnotehyper.sty}{\usepackage{footnotehyper}}{\usepackage{footnote}}
\makesavenoteenv{longtable}
\usepackage{graphicx}
\makeatletter
\def\maxwidth{\ifdim\Gin@nat@width>\linewidth\linewidth\else\Gin@nat@width\fi}
\def\maxheight{\ifdim\Gin@nat@height>\textheight\textheight\else\Gin@nat@height\fi}
\makeatother
% Scale images if necessary, so that they will not overflow the page
% margins by default, and it is still possible to overwrite the defaults
% using explicit options in \includegraphics[width, height, ...]{}
\setkeys{Gin}{width=\maxwidth,height=\maxheight,keepaspectratio}
% Set default figure placement to htbp
\makeatletter
\def\fps@figure{htbp}
\makeatother
\setlength{\emergencystretch}{3em} % prevent overfull lines
\providecommand{\tightlist}{%
  \setlength{\itemsep}{0pt}\setlength{\parskip}{0pt}}
\setcounter{secnumdepth}{5}
\usepackage{float}
\floatplacement{figure}{H}
\usepackage{booktabs}
\usepackage{longtable}
\usepackage{array}
\usepackage{multirow}
\usepackage{wrapfig}
\usepackage{float}
\usepackage{colortbl}
\usepackage{pdflscape}
\usepackage{tabu}
\usepackage{threeparttable}
\usepackage{threeparttablex}
\usepackage[normalem]{ulem}
\usepackage{makecell}
\usepackage{xcolor}
\ifluatex
  \usepackage{selnolig}  % disable illegal ligatures
\fi

\title{R Markdown Output Type Test}
\author{R. M. D. Trickster}
\date{2021-01-24 18:59}

\begin{document}
\maketitle

{
\setcounter{tocdepth}{2}
\tableofcontents
}
\hypertarget{output-format-testing}{%
\section{Output format testing}\label{output-format-testing}}

This document provides an example of using a single Rmd file to generate output with different content based on the output document type.

The envelope-pushing examples below are complicated, and in many cases require generating strings that are rendered using the structure:

The examples are unlikely--it is hard to imagine a case for which a single Rmd file should generate completely different output. But the examples should serve to demonstrate how different output can be included based on the type of output file.

If this is an HTML document, the following apply:

\begin{enumerate}
\def\labelenumi{\arabic{enumi}.}
\tightlist
\item
  Text below indicates ``This is HTML output and shows a Leaflet map.''
\item
  A Leaflet map is shown, along with a caption and cross-reference
\item
  Code chunks are displayed inline, ``hidden'' by default.
\item
  There is no source code at the end of the document.
\end{enumerate}

If this is PDF output, the following apply:

\begin{enumerate}
\def\labelenumi{\arabic{enumi}.}
\tightlist
\item
  Text below indicates ``This is PDF output and therefore a Leaflet map cannot be added.''
\item
  The Poisson distribution equation is shown, with an equation number and cross reference.
\item
  Code chunks are not shown inline.
\item
  Source code is printed at the end of the document.
\end{enumerate}

This is PDF output and therefore a Leaflet map cannot be added.

See the Poisson distribution (1) for details
\begin{equation}
      P\left( x \right) = \frac{{e^{ - \lambda } \lambda ^x }}{{x!}}
    \end{equation}

\hypertarget{source-code}{%
\section{Source code}\label{source-code}}

\begin{verbatim}
---
title: "R Markdown Output Type Test"
author: "R. M. D. Trickster"
date: '`r format(Sys.time(), "%Y-%m-%d %H:%M")`'
header-includes: #allows you to add in your own Latex packages
- \usepackage{float} #use the 'float' package
- \floatplacement{figure}{H} #make every figure with caption = h
output:
bookdown::pdf_document2:
number_sections: true
toc: true
fig_cap: yes
keep_tex: yes
bookdown::html_document2:
number_sections: true
self_contained: true
code_folding: hide
toc: true
toc_float:
collapsed: true
smooth_scroll: false
---

```{r setup, include=FALSE}
library(sf)
library(leaflet)
library(kableExtra)
library(magrittr)
library(knitr)
library(tidyverse)
library(captioner)

# captions
eqn_captions <- captioner::captioner(prefix="")
fig_captions <- captioner::captioner(prefix="Figure")


# test for output type; if HTML, include code; if PDF do not include code
if(is_html_output()){
knitr::opts_chunk$set(echo = TRUE)
} else {
knitr::opts_chunk$set(echo = FALSE)
}
```

# Output format testing
This document provides an example of using a single Rmd file to generate output
    with different content based on the output document type.

The envelope-pushing examples below are complicated, and in many cases require
    generating strings that are rendered using the structure:

<code><pre>```{r results='asis'}
cat("
some R code that generates some output
")
```</pre></code>

The examples are unlikely--it is hard to imagine a case for which a single Rmd
    file should generate completely different output. But the examples should
    serve to demonstrate how different output can be included based on the type
    of output file.

If this is an HTML document, the following apply:

1. Text below indicates "This is HTML output and shows a Leaflet map."
1. A Leaflet map is shown, along with a caption and cross-reference
1. Code chunks are displayed inline, "hidden" by default.
1. There is no source code at the end of the document.

If this is PDF output, the following apply:

1. Text below indicates "This is PDF output and therefore a Leaflet map cannot
    be added."
1. The Poisson distribution equation is shown, with an equation number and cross
    reference.
1. Code chunks are not shown inline.
1. Source code is printed at the end of the document.

```{r, results='asis'}
if(is_html_output()){
foo <- ""
cat("Because this is HTML output, a Leaflet map could be included!\n\n")
cat("## A Leaflet Map\n")
}
```

```{r}
if(is_html_output()){
# the Space Needle
snxy <- data.frame(name = "Space Needle", x = -122.3493, y = 47.6205)
space_needle <- st_as_sf(snxy, coords = c("x", "y"), crs = 4326)

# a leaflet map
m <- leaflet() %>%
addTiles() %>%
addCircleMarkers(data = space_needle)
m
}
```

```{r, results='asis'}
if(is_html_output()){
# make and print the caption
# this creates a string object from the citation, and drops all white space
cap <- fig_captions(name = "Leaflet", caption = "A Leaflet map centered at the
    Space Needle\n\n")
cat(cap)

cite <- fig_captions(name = "Leaflet", display = "cite")

# This adds the text
cat(sprintf("See the Leaflet map (%s) for details.", cite))
}
```

```{r, results='asis'}
# if this is LaTeX output then print the Poisson distribution and use an
    equation number and cross-reference
if(is_latex_output()){
cat("This is PDF output and therefore a Leaflet map cannot be added.\n\n")

# an equation caption
eqn_captions(name = "poisson", caption = "")

# this creates a string object from the caption, and drops all white space
cite <- eqn_captions(name = "poisson", display = "cite") %>%
str_remove_all(pattern = "\\s")

# This adds the text
cat(sprintf("See the Poisson distribution (%s) for details", cite))

# `cat' to print out an equation
cat("
\\begin{equation}
P\\left( x \\right) = \\frac{{e^{ - \\lambda } \\lambda ^x }}{{x!}}\
\\end{equation}
")
}
```

```{r results='asis'}
# only print the source code if this is LaTeX output
if(is_latex_output()){
cat("# Source code\n\n")
}
```

```{r comment='', }
# only print the source code if this is LaTeX output
if(is_latex_output()){
srccode <- readLines("output_type_test.Rmd") %>%
str_wrap(width = 80, exdent = 4)
cat(srccode, sep = '\n')
}
```
\end{verbatim}

\end{document}
